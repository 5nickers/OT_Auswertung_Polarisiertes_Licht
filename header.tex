\documentclass[12pt,a4paper]{scrartcl} %scrartcl berücksichtigt deutsche Formatierungen besser als article
\usepackage[T1]{fontenc}
\usepackage[a4paper, left=3cm, right=2cm, top=2cm, bottom=2cm]{geometry}
\usepackage[activate]{pdfcprot}
\usepackage[ngerman]{babel} %n = neue deutsche Rechtschreibung
\usepackage[parfill]{parskip}
\usepackage[utf8]{inputenc}
\usepackage{kurier}
\usepackage{amsmath}
\usepackage{amssymb}
\usepackage{xcolor}
\usepackage{epstopdf}
%\usepackage{txfonts} % Paket für Word-ähnliche Schriftarten Arial (serifenlos) oder Times New Roman (mit Serifen)
\usepackage{fancyhdr}
\usepackage{graphicx}
\usepackage{prettyref}
\usepackage{hyperref} %Internet-/E-Mail-Verknüpfungen
\usepackage{eurosym}
\usepackage{setspace}
\usepackage{units} % Einheitenpaket
\usepackage{eso-pic,graphicx}
\usepackage{icomma} % nach Komma kein Leerzeichen
\usepackage{textcomp} % fuer spezielle Zeichen, wie \textmu  \textcelsius \textcopyright  \texteuro  \textcent  \textdollar  \textnumero
\usepackage{siunitx}
	%\sisetup{locale = DE,mode = text}         Bruch alles in einer Zeile
	%\sisetup{locale = DE,per-mode = symbol}	  Bruch mit /
	\sisetup{locale = DE,per-mode = fraction} %Bruch als richtiger Bruch
	%im Document:
	%\si{Zahl oder Einheit}
	%\SI{Zahl}{Einheit}		\SI{1,03}{\volt\per\square\metre}
\usepackage{floatrow} % um Bilder mit seitlichem Text zu versehen
%\usepackage[capbesideposition=inside, facing=yes,capbesidesep=quad]{floatrow}


% fuer Grafiken
%\usepackage[pdftex]{color,graphicx}
%\graphicspath{{figures/}}
\DeclareGraphicsExtensions{.png,.jpg,.pdf}
\usepackage{float} % Anwendung: „[H]“ Bild unbedingt an der Stelle, wo \includegraphics steht.
\usepackage{eso-pic,graphicx}
\usepackage{picinpar} % Bilder werden vom Text im folgenden Format umflossen: \begin{window} [#zeilen-vor, position, graphik, titel] Text(der die Grafik umfließen soll) \end{window}

\usepackage{pgfplots} % u.a. für Diagramme
\pgfplotsset{width=14cm,compat=newest}



\definecolor{darkblue}{rgb}{0,0,.5}
\hypersetup{colorlinks=true, breaklinks=false, linkcolor=black, menucolor=black, urlcolor=darkblue} % rote Standardumrandung neu definiert

%\renewcommand{\familydefault}{\sfdefault}  % Textstil serifenlos

\setlength{\columnsep}{2cm}
\newcommand{\fehlt}{\textcolor{red}{Hier fehlen noch Inhalte.}}
\renewcommand{\d}{\, \mathrm d}
\newcommand{\p}{\, \partial}
\newcommand{\dd}[1]{\item[#1] \hfill \\}

\newcommand{\themodul}{Praktikum Optische Technologien}
\newcommand{\thetutor}{Protokoll Versuch polarisiertes Licht}

% Kopf- und Fußzeile
\pagestyle{fancy}
\fancyhead[L]{\footnotesize{M. Nonhoff, C. Hansen, J. Ehlert}}
\chead{\thepage}
\rhead{\footnotesize{\thetutor}}
\lfoot{}
\cfoot{}
\rfoot{}

% Überschriftaufbau
\title{\themodul{}, \\ \thetutor}
\author{Marko Nonhoff, Christoph Hansen, Jannik Ehlert \\
	\small \href{chris@university-material.de}{chris@university-material.de} }
\date{} %wenn leer, dann heutiges Datum